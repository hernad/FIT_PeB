\documentclass[times, utf8, seminar]{fit}

%\batchmode
%\usepackage{booktabs}
\usepackage{listings}
\usepackage{longtable}
\usepackage{xcolor}
\usepackage{float}
\usepackage{enumitem}
\usepackage{hyperref}
\usepackage{enumerate}
\usepackage{graphicx}
\usepackage{etoolbox}
\usepackage{datetime}
\usepackage{needspace}
\usepackage{titlesec}
%\titleformat{\chapter}[display]{\normalfont\huge\bfseries}{\chaptertitlename\ \thechapter}{20pt}{\Huge}

\begin{document}
\widowpenalty=300
\clubpenalty=300


% this alters "before" spacing (the second length argument) to 0
%\titlespacing*{\chapter}{0pt}{0pt}{40pt}

% this changes "before" spacing back to its default of 50pt
%\titlespacing*{\chapter}{0pt}{50pt}{40pt}}

%\titlespacing*{\chapter}{0pt}{-50pt}{18pt}
%\titleformat{\chapter}[display]{\normalfont\huge\bfseries}{\chaptertitlename\ \thechapter}{20pt}{\Huge}

\title{OSS web e-commerce rješenje sa sistemom preporuke}

\author{Ernad Husremović}
\brindex{DL 2792}
\verzija {1.0.1}

\mentor{mr. Haris Balta}

\maketitle

\tableofcontents

%\listoftables
%\listoffigures
\newpage

% abstract begin
%\begin{abstract}
%
%To be done 
%
%\keywords{open source software, OSS, Bosna i Hercegovina}
%\end{abstract}

% abstract end

\chapter{Uvod}
\vspace*{-0.7cm}

Cilj ovog seminarskog rada je implementiracija webshop rješenja čije su komponente isključivo open source (nadalje OSS) software 

Firma "bring.out", implementator sistema, je bosanski OSS IT provajder.

Kompletan portfolio firme je baziran na OSS-u, tako da i ovo rješenje treba zadovoljiti kriterije otvorenosti izvornog koda.

\chapter{Analiza dostupnih OSS komponenti za gradnju rješenja}
\vspace*{-0.7cm}

\section{Sistemi preporuke}

Najpoznatiji OSS sistemi preporuke su:

\begin{itemize}
  \item easyrec\footnote{\url{http://easyrec.org/}}
  \item Apache mahout\footnote{\url{http://mahout.apache.org/}}
  \item Lenskit recommender framework\footnote{\url{http://lenskit.grouplens.org/}}
\end{itemize}

\emph{Apache mahout} je projekat koji ima snažnu reputaciju kvalitetnog i fleksibilnog "recommendation engine"-a 

Postoji visoka integracija sa drugim projektom popularnim "Apache fondacije" - hadoop-om\footnote{\url{http://hadoop.apache.org}} koji se koristi u za distribuiranu obradu "big data"\footnote{velike količine podataka, najčešće korišteno u  rješenjima "data-mining" sistema}.

Sva tri pomenuta rješenja su implementirana u programskom jeziku Java.

\emph{Lenskit} je najmlađi projekat, ali ima veoma aktivnu zajednicu korisnika i programera (eng. "community").

\emph{Easyrec} je započeo kao komercijalni projekat austrijske firme "Studio Smart Agent Technologies". Naknadno je objavljen kao OSS pod GPLv3 licencom. 
Dobio je više nagrada za inovaciju na području web-a i multimedije\footnote{\url{http://en.wikipedia.org/wiki/Easyrec}}. 
Easyrec je java aplikacija koja sa webshop rješenjima komunicira putem web servisa (REST ili SOAP), tako da je moguća integracija sa bilo kojim third-party sistemom.
Sadrži bogat i kvalitetan web administrativni interfejs, što ga definitivno izdvaja u odnosu na konkurentska rješenja.

\begin{figure}[H]
\centering
\includegraphics[width=10cm]{img/easyrec_1_tenant.png}
\caption{Easyrec web interfejs}
\end{figure}

\section{Web shop rješenje}

Postoji niz OSS opcija za webshop komponentu. Međutim, sa stanovišta proširivosti, "drupal"{\footnote{\url{http://www.drupal.org}} bazirana rješenja su najpopularnija.

\subsection{Drupal web content framework}

Drupal je "web content framework" napisan u programskom jeziku PHP. 

Često se navodi i da je drupal "CMS"{\footnote{CMS - Content Management System}} software, međutim \emph{framework} je 
definitivno bolji opis drupal-a.

\begin{figure}[H]
\centering
\includegraphics[width=10cm]{img/drupal_add_content.png}\hfill
\caption{Drupal web interfejs dodavanje sadržaja (content)}
\end{figure}

\begin{figure}[H]
\centering
\includegraphics[width=10cm]{img/drupal_block_structure.png}\hfill
\caption{Drupal web interfejs - podešenje strukture prikaza (blocks)}
\end{figure}


%\begin{center}
%\emph{\large{Freedom to create, distribute, and use open source software (OSS).}}
%\end{center}

\subsection{OSS E-commerce rješenja}

Na tržištu postoji više OSS aktivnih drupal e-commerce projekata:

\begin{itemize}
\ubercart  
\item drupal commerce\footnote{\url{http://www.drupalcommerce.org}} 
\item ubercart\footnote{\url{http://www.ubercart.org}}
\end{itemize}

"Drupal comerce" je relativno mlad projekat, ali je vrlo brzo stekao vidljivost na tražištu OSS e-commerce rješenja.

Započeli su ga bivši developer "ubercart"-a, koji je opet prvo "ozbiljno" OSS e-commerce rješenje.

Web stranice prokejkta ukazuju da se radi o ozbiljnom projektu sa jakom finansijskom podrškom:

\begin{figure}[H]
\centering
\includegraphics[width=4.5cm]{img/drupalcommerce_web.png}\hfill
\includegraphics[width=4.5cm]{img/drupalcommerce_web_2.png}\hfill
\includegraphics[width=4.5cm]{img/drupalcommerce_web_3.png} 
\caption{Drupal commerce web site}
\end{figure}

Filozofija razvoja drupal commerce-a je slična razvoju samog drupala. Drupalcommerce je core sistem za nove e-commerce distribucije koje su bazirane na njemu.
Source code je GPLv2 licenciran kao i sam drupal.

\emph{Ubercart} je svojevrsni "veteran" drupal baziranih e-commerce rješenja. Ubercart v2.x (baziran na drupal 6.x) je vrlo zastupljen e-commerce sistem.

\subsection{Integracijska komponenta (e-commerce <-> sistem preporuke)}

Posljednja komponenta u konstrukciji zadanog sistema je integracijska komponenta e-commerce <-> recommender.

Ubercart ima riješenu integraciju sa easyrec-om kao poseban drupal projekat{\footnote{\url{http://drupal.org/project/easyrec\_for\_ubercart}} 

\section{Odabir komponenti}

Easyrec se svojom jednostavnošću i fleksibilnošću odmah nametnuo kao dobar izbor za komponentu koja će biti sistem preporuke. 
Kako za njega postoji i integracijska komponenta sa ubercart-om konačan izbor sistema je:
\begin{enumerate}
  \item drupal 7.x
  \item ubercart 3.x
  \item easycart 0.97
  \item easyrec\_for\_ubercart
\end{enumerate}

%\begin{quotation}
%\emph{In 2009, more than four out of 10 software programs installed on personal computers around the world were stolen, with a commercial value of more than \$51 billion. Unauthorized software can manifest in otherwise legal businesses that buy too few software licenses, or cover criminal enterprises that sell counterfeit copies of software programs at cut-rate prices, online or offline.}

%\emph{However, the impact of software piracy goes beyond revenues lost to the software industry, starving local software distributors and service providers of spending that creates jobs and generates much-needed tax revenues for governments around the world.}\citep{bsapiracyimpact}
%\end{quotation}\\

%\chapter{OSS in Bosnia and Herzegovina}

\chapter{Glavne funkcije dobijenog rješenja}
\vspace*{-0.7cm}

\section{Definisanje artikla - proizvoda}

\begin{figure}[H]
\centering
\includegraphics[width=10cm]{img/drupal_add_content_1.png}
\caption{Dodavanje novog proizvoda / 1}
\end{figure}


\begin{figure}[H]
\centering
\includegraphics[width=10cm]{img/drupal_add_content_2.png}
\caption{Dodavanje novog proizvoda / 2}
\end{figure}

\begin{figure}[H]
\centering
\includegraphics[width=10cm]{img/drupal_add_content_3.png}
\caption{Dodavanje novog proizvoda / 3}
\end{figure}

Na kraju, proizvod se prikazuje na naslovnoj stranici našeg web-shopa:

\begin{figure}[H]
\centering
\includegraphics[width=10cm]{img/drupal_add_content_4.png}
\caption{Dodavanje novog proizvoda / 4}
\end{figure}

\section{Prodaja artikala}

Prodaja artikala unutar uberchart webshop rješenja je krajnje jednostavna. Klik na dugme proizvoda i korpa kupca se puni:

\begin{figure}[H]
\centering
\includegraphics[width=14cm]{img/prodaja.png}
\caption{Potrošačka korpa "kupca"}
\end{figure}

Potvrdom narudžbe se prodaja registruje:

\begin{figure}[H]
\centering
\includegraphics[width=14cm]{img/review_order.png}
\caption{Potvrdom narudžbe se registruje prodaja kako u web-shopu tako i u easyrec-u (kao "buy" akcija)}
\end{figure}

\section{Najposjećeniji artikli (pregledi i kupnja)}

\begin{figure}[H]
\centering
\includegraphics[width=10cm]{img/pregled_rezultata_preporuke.png}
\caption{Informacije o artiklima koji se najviše gledaju i kupuju}
\end{figure}

Međutim, gornji set informacija odnosi se na kompletan webshop, što znači da su gornje informacije identične za sve korisnike.

\section{Korisnički orjentisane preporuke}

Sistem preporuke kod pregleda pojedinačnog artikla treba davati informacije uzimajući u obzir kontekst - treba dati preporuke u kontekstu trenutnog korisnika i artikla koji se gleda.

Samo tako preporuke tipa "drugi najviše kupuju" ili "drugi najviše gledaju" mogu dati pravi efekat" - korisnik će dobiti samo relevantne informacije.

Jednu takvu situacija se može simulirati na sljedeći način:
\begin{itemize}
  \item Kao admin unesemo novi artikal (P10)
  \item Zatim korisnici test1, test2, test3 izvrše pregled tog novog artikla
  \item Pošto easyrec statistiku ažurira svaka 24h, ručno izvršimo ažuriranje statistike u administracijskoj konzoli
  \item prijavimo se kao korisnik test4, otiđemo na pregled nekog starog artikla - dobijamo informaciju da su drugi gledali artikal P10. 
           To je upravo informacija koju korisnik treba.
  \item Kada se prijavimo kao anonimni korisnik, dobijamo nešto sasvim drugo - čitavu listu artikala koje sistem preporučuje
  \item Korisnicima koji su već pregledali novi artikal se on više ne nudi\footnote{Treba znati da se ovo konfigurisati u easyrec-u, tako da se artikli ponovo preporučuju korisniku}
\end{itemize}

Evo prikaza ekrana za gore opisane situacije. Korisnik test4:

\begin{figure}[H]
\centering
\includegraphics[width=10cm]{img/preporuka_pojavio_se_novi_artikal_p10.png}
\caption{Sistem preporuke informiše korisnika o novom artiklu, koji su drugi korisnici gledali}
\end{figure}

Anonimni korisnik:

\begin{figure}[H]
\centering
\includegraphics[width=10cm]{img/preporuka_anonimni_korisnik.png}
\caption{Anonimni korisnik dobija čitavu listu artikala kao prijedlog da ih pregleda}
\end{figure}

\chapter {Sistem preporuke}
\vspace*{-0.7cm}

\section{Sistemi preporuke, općenito}

Količina informacije koje internet korisnici dobijaju se stalno povećava. To za posljedicu ima sve teže i teže nalaženje relevatnih informacija. 
Zato su tehnike kojima se obezbjeđuje da korisnici dobijaju samo \emph{relevatne} informacija dobile na velikoj važnosti. 
Set algoritama koji nastoje riješiti taj problem nazivaju se sistemi za kolaborativno filterisanje, odnosno sistemi preporuke (eng. CF - Collaborative Filtering recommender systems). 
CF algoritmi pohranjuju akcije koje pojedini korsinici vrše nad artiklima.
Ovi podaci se onda koriste da se generišu grupacije ("susjedstva") sličnih artikala ili korisnika sa pretpostavkom
da korisnik želi vidjeti iste stvari (proizvode, usluge) kao i korisnici koji su u bliskom "susjedstvu" sa njim po pitanju ličnih preferencija\citep{bac_wien}

\section{Easyrec sistem preporuke}

Easyrec\footnote{\url{http://en.wikipedia.org/wiki/Easyrec}} je cjelovito rješenje sistema preporuke.

\begin{figure}[H]
\centering
\includegraphics[width=10cm]{img/Easyrec_architecture.png}
\caption{Arhitektura easyrec aplikacije (izvor wikipedija)}
\end{figure}

On sistemom plugin-ova može kombinovati više algoritama za ragniranje artikala, te na osnovu njih davati različite preporuke korisnicima sistema.

Easyrec ima sopstvenu bazu artikala prema kojoj vrši proračune. Ona prima podatke o akacijama korisnika nad pojedinim artiklima unutar e-commerce sistema, 
procesira ih putem algoritama preporuke, te korisnicima vraća rezultat u obliku rang-lista po više različitih kriterija.

Na njihovoj web stranici\footnote{\url{http://easyrec.org}} stoji:

\begin{quotation}
easyrec is:
\begin{itemize}
\item easy to use - personalize your application within minutes
\item easy to integrate - due to plugins, a REST API or placing javascript codesnippets in your web pages.
\item easy to scale - due to distributed architecture
\item easy to maintain - with the included management application
\item easyrec is OPEN SOURCE and FREE!
\end{itemize}
\end{quotation}

\subsection{Easyrec plugin sistem} 

Aplikacija dolazi sa dva ugrađena plugin-a:
\begin{itemize}
  \item Slope one
  \item ARM (Association Rule Miner)\footnote{baziran na Apriori algoritmu od R. Agrawal 1994}
\end{itemize}

\begin{figure}[H]
\centering
\includegraphics[width=12cm]{img/easyrec_10_plugins.png}
\caption{Algoritmi koji se primjenjuju su easyrec plugin-ovi}
\end{figure}

\subsection{Statistika proračuna, sa prikazom primjenjenih algoritama}

\begin{figure}[H]
\centering
\includegraphics[width=12cm]{img/easyrec_statistika_1.png}
\caption{eaysrec statistika / 1}
\end{figure}

\begin{figure}[H]
\centering
\includegraphics[width=12cm]{img/easyrec_statistika_2.png}
\caption{eaysrec statistika / 2}
\end{figure}

\begin{figure}[H]
\centering
\includegraphics[width=12cm]{img/easyrec_statistika_3.png}
\caption{eaysrec statistika / 3}
\end{figure}

\begin{figure}[H]
\centering
\includegraphics[width=12cm]{img/easyrec_statistika_4.png}
\caption{eaysrec statistika / 4}
\end{figure}

\subsection{Data model}

\begin{figure}[H]
\centering
\includegraphics[width=14cm]{img/easyrec_data_model.png}
\caption{eaysrec data model\citep{bac_wien}}
\end{figure}

\subsection{Tenant}

"Tenant" označava aplikaciju kojoj easyrec pruža usluge. U našem slučaju to je naša web-shop aplikacija. 
\begin{figure}[H]
\centering
\includegraphics[width=12cm]{img/easyrec_1_tenant.png}
\caption{Easyrec "tenant" ubercart2}
\end{figure}

\subsection{Items}

Items stavke predstavljaju glavni predmet analize sistema preporuke.

\begin{figure}[H]
\centering
\includegraphics[width=12cm]{img/easyrec_2_item.png}
\caption{Easyrec "items" koje "tenant" ubercart2 bilježi}
\end{figure}

\begin{figure}[H]
\centering
\includegraphics[width=12cm]{img/easyrec_3_item.png}
\caption{Easyrec prikaz jednog "item"-a}
\end{figure}

\subsection{Pregledi dostupni za "items"}

Slijedi pregledi podataka koje easyrec priprema "out-of-box":

\begin{figure}[H]
\centering
\includegraphics[width=12cm]{img/easyrec_4_top_ranked.png}
\caption{Easyrec "top ranked" artikli}
\end{figure}

\begin{figure}[H]
\centering
\includegraphics[width=12cm]{img/easyrec_5_hot_recommend.png}
\caption{Easyrec preporučeni ("hot recommend") artikli}
\end{figure}

\begin{figure}[H]
\centering
\includegraphics[width=12cm]{img/easyrec_6_item_types.png}
\caption{Easyrec tipovi "itema" (stavki koje ocjenjujemo i preporučujemo) }
\end{figure}

\begin{figure}[H]
\centering
\includegraphics[width=12cm]{img/easyrec_7_item_actions.png}
\caption{Easyrec akcije koje se bilježe za stavke}
\end{figure}

\begin{figure}[H]
\centering
\includegraphics[width=12cm]{img/easyrec_8_item_assoc.png}
\caption{Easyrec asocijacije između stavki}
\end{figure}

\chapter{Uočeni problemi}

Testiranje ovog sistema je veoma teško realizovati. Manuelni način koji je korišten za pripremu ovog rada (napraviti više 5 usera, 10 artikala) je veoma zamoran, a uz to rezultati testiranja su nepouzdani.

Tom dijeli bi se u slučaju realizacije produkcijskog sistema morala posvetiti velika pažanja.

Naime, ako bi korisnicima isporučili aplikaciju sa sistemom preporuke koja "laže" korisnike, napravili veću štetu nego korist.

Uočio sam da je u nekim situacijama easyrec "gutao" stavke koje mu je webshop davao. To se redovno dešavalo ako bi se slala veća količina podataka. 

Postojeće rješenje easyrec\_uberchart integracije, po svemu sudeću, ima dosta nedostataka. Tako recimo prilikom prodaje 50 x artikla Proizvod10 klijent šalje serveru 50 zahtjeva.

Ni u "easyrec" javascript API dokumentaciji nisam našao način kako da se informacija proslijedi jednim "requestom".

\chapter{Zaključak}

I pored navedenih problema, dobijeno rješenje predstavlja odličnu osnovu za realizaciju produkcijskog e-commerce rješenja sa sistemom preporuke.

Dostupna OSS rješenja sistema preporuke pružaju mogućnost da kreatori e-commerce aplikacija, bez velikih invensticija u razvoj i testiranje algoritama sistema preporuke, naprave najsavremenija e-commerce rješenja.

Prednosti opensource software-a u ovakvim projektima posebno dolaze do izražaja. 
Naime, implementacija kvalitetnog "recommender" sistema iz početka bi kod većine projekata bila stavka koja se ne bi mogla uklopiti u budžet.

Iako se ne treba zanemariti napor da se izvrši kvalitetna integracija odgovarajuće OSS "recommender" komponente, ipak je ta opcija kako vremenski, tako i finansijski višestruko bolji izbor.

Tokom izrade seminarskog rada sam napravio prve korake sa "drupal"-om. 

Radi se o iznimno moćnom razvojnom frameworku. Treba ipak naglasiti da je potrebno je dosta vremena da se tim okruženjem ovlada.

Na kraju svega, mislim da je moj izbor komponenti bio dobar.

U nekom novom radnom ciklusu na ovu temu, svakako bi bilo dobro istražiti mogućnosti "drupalcommerce"-a za web-shop i "mahout/taste" framework u ulozi "recommender engine"-a.

Recommender sistemi su bez dvojbe sjajna stvar u e-commerce sistemima. Tačnije u svim sistemima koji rade sa velikim brojek korisnika.

Ako se dobro realiziraju, korisnici će sigurno prepoznati i cijeniti postojanje sistema preporuke.

% -------------------------------------------------
\bibliography{literatura}
\bibliographystyle{fit}

% -------------------------------------------------
\appendix

\chapter{Instalacija}
\vspace*{-0.7cm}
\setlength{\parindent}{0cm}
%\setlength{\parindent}{default}

\section{Bazno okruženje}

\subsection{XAMPP PHP, Mysql database server}
Drupal instaliramo unutar  XAMPP stack-a{\footnote{\url http://www.apachefriends.org/en/xampp.html} za Mac OS X 1.7.3 (MySQL 5.1.x, PHP 5.3.x)

\lstset{
  language=bash,
  backgroundcolor=\color{gray!25},
  basicstyle=\ttfamily \footnotesize,
  breaklines=true,
  prebreak=\raisebox{0ex}[0ex][0ex] \hookleftarrow,
  columns=fullflexible
}

%\lstset{
%}
%\lstset{postbreak=\raisebox{0ex}[0ex][0ex] \rcurvearrowse\space}
%\lstset{breaklines=true, breakatwhitespace=true}
%\lstset{numbers=left, numberstyle=\scriptsize}

Apache web server web root:
\begin{lstlisting}
/Applications/XAMPP/htdocs
\end{lstlisting}

\subsection{Easyrec web aplikacija}

Easyrec je stanardna J2EE\footnote{java enterprise edition} web aplikacija (WAR), tako da za njegovu instalaciju trebamo J2EE aplikacijski server. 

Preporučeni server je tomcat. Instaliramo tomcat, kopiramo easyrec-web.war (ver. 0.97)\footnote{download sa sourceforge repozitorija \url{http://sourceforge.net/projects/easyrec/files/}} u deployment direktorij tomcat servera i na kraju pokrećemo server:

\begin{lstlisting}

1) ~/FIT/PeB/apache-tomcat-7.0.30$ cp ~/Downloads/easy-rec.war webapps/

2) ~/FIT/PeB/apache-tomcat-7.0.30/bin\$ ./catalina.sh start

> Using CATALINA_BASE:   /Users/hernad/FIT/PeB/apache-tomcat-7.0.30
> Using CATALINA_HOME:   /Users/hernad/FIT/PeB/apache-tomcat-7.0.30
> Using CATALINA_TMPDIR: /Users/hernad/FIT/PeB/apache-tomcat-7.0.30/temp
> Using JRE_HOME:        /System/Library/Frameworks/JavaVM.framework/Versions/CurrentJDK/Home
> Using CLASSPATH:       /Users/hernad/FIT/PeB/apache-tomcat-7.0.30/bin/bootstrap.jar:/Users/hernad/FIT/PeB/apache-tomcat-7.0.30/bin/tomcat-juli.jar
\end{lstlisting}

Pokrećemo u browseru 

\begin{lstlisting}
http://localhost:8080/easyrec-web
\end{lstlisting}

prolazimo kroz jednostavni install proces koji u easyrec bazu\footnote{koju ranije trebamo kreirati, vidi donji korak "Kreiranje easyrec baze"} 

Na kraju dobijamo dolazimo na login našeg sistema preporuke:

\subsection{Kreiranje drupal i easyrec baza}

\begin{lstlisting}
/Applications/XAMPP/xamppfiles$ bin/mysql -u root
mysql> create database ubercart3;

> Query OK, 1 row affected (0.07 sec)

mysql> create database easyrec;

mysql> quit

\end{lstlisting}

\subsection{Drupal}

Drupal core 7.1.15 sa svim potrebni podmodulima nalazi se na hernad github repozitoriju. 

Izvršimo kloniranje repozitorija i njegovih podmodula:

\begin{lstlisting}
1) /Applications/XAMPP/htdocs$ git clone git@github.com:hernad/drupal.git

>  Cloning into 'drupal'...

/Applications/XAMPP/htdocs$ cd drupal

2) /Applications/XAMPP/htdocs/drupal$ git submodule init 

> Submodule 'sites/all/modules/easyrec_for_ubercart' (git@github.com:hernad/easyrec_for_ubercart.git) registered for path 'sites/all/modules/easyrec_for_ubercart'
> Submodule 'sites/all/modules/jcarousel' (git@github.com:hernad/jcarousel.git) registered for path 'sites/all/modules/jcarousel'
> Submodule 'sites/all/modules/ubercart' (git@github.com:hernad/ubercart.git) registered for path 'sites/all/modules/ubercart'

3) /Applications/XAMPP/htdocs/drupal$ git submodule update

> Cloning into 'sites/all/modules/easyrec_for_ubercart'...
> ...
> Submodule path 'sites/all/modules/easyrec_for_ubercart': checked out '26a96e622be0acee012c4cff85b3e8e4f87cd5e5'
> Cloning into 'sites/all/modules/jcarousel'...
> ....
> Submodule path 'sites/all/modules/jcarousel': checked out 'ef51d00eae77e354c771cf595621971f5f250ac2'
> Cloning into 'sites/all/modules/ubercart'...
> ...
> Submodule path 'sites/all/modules/ubercart': checked out 'e5a4154d8a9fdf6483fabd7e851798f3c88fd623'
\end{lstlisting}

Podešenje drupal config fajlova:

\begin{lstlisting}
/Applications/XAMPP/htdocs/drupal/sites$ cp example.sites.php sites.php
/Applications/XAMPP/htdocs/drupal/sites$ chmod a+w sites.php
/Applications/XAMPP/htdocs/drupal/sites$ mkdir -p default/files
/Applications/XAMPP/htdocs/drupal/sites$ chmod a+w default/files
/Applications/XAMPP/htdocs/drupal/sites$ cd default
/Applications/XAMPP/htdocs/drupal/sites/default$ cp default.settings.php  settings.php
/Applications/XAMPP/htdocs/drupal/sites/default$ chmod a+w settings.php
/Applications/XAMPP/htdocs/drupal/sites/default$ cd ..
\end{lstlisting}

\begin{figure}[H]
\centering
\includegraphics[width=10cm]{img/drupal_install_1.png}
\caption{Drupal install step 1}
\end{figure}

\begin{figure}[H]
\centering
\includegraphics[width=10cm]{img/drupal_install_2.png}
\caption{Drupal install step 2}
\end{figure}

\begin{figure}[H]
\centering
\includegraphics[width=10cm]{img/drupal_install_3.png}
\caption{Drupal install step 3}
\end{figure}

\begin{figure}[H]
\centering
\includegraphics[width=10cm]{img/drupal_install_4.png}
\caption{Drupal install step 4}
\end{figure}

\begin{figure}[H]
\centering
\includegraphics[width=9cm]{img/drupal_install_5.png}
\caption{Drupal install step 5}
\end{figure}

\begin{figure}[H]
\centering
\includegraphics[width=9cm]{img/drupal_install_6.png}
\caption{Drupal install step 6}
\end{figure}


\begin{figure}[H]
\centering
\includegraphics[width=12cm]{img/drupal_welcome.png}
\caption{Drupal install - napokon "welcome" :)}
\end{figure}

\subsection{Aktivacija drupal modula za ubercart i easyrec}

\begin{enumerate}
\item locale
\item chaost-tools
\item easyrec
\item rules
\item cart
\item store
\item order
\item product
\item views
\item views-ui
\end{enumerate}

\subsection{Podešenje drupal blokova za prikaz ubercart i easyrec elemenata}

Prikaz sadržaja na drupal stranicama se podešava unutar admin interfejsa.

Mi želimo na "featured" sekciju (vrh web stranice) staviti prikaz najviše kupovanih i najviše gledanih artikala u toku dana\footnote{Podešavamo dnevnu statistiku u testnom okruženju zato što se ona najprije ažurira}

\begin{figure}[H]
\centering
\includegraphics[width=12cm]{img/easyrec_ubercart_blocks_config_1.png}
\caption{Drupal block config}
\end{figure}

Ovako podešavamo prikaz najviše kupovanih artikala:

\begin{figure}[H]
\centering
\includegraphics[width=12cm]{img/easyrec_ubercart_blocks_config_2.png}
\caption{Konfiguracija drupal blokova - prikaz sistema preporuke}
\end{figure}

Analogno se podešava i prikaz artikala koje su korisnici najviše pregledali.

\subsection{Podešenje komunikacije ubercart - easyrec server}

Dodajmo jedan "tenant" u easyrec\footnote{Inače tenant je u prevodu sa engleskog "stanar". U kontekstu easyrec-a, tenant je e-commerce aplikacije koju monitorišemo.}

\begin{figure}[H]
\centering
\includegraphics[width=12cm]{img/integ_easyrec.png}
\caption{Kreiranje novog tenanta ubercart3}
\end{figure}

Povezivanje drupala sa easyrec tenant-om "ubercart3":

\begin{figure}[H]
\centering
\includegraphics[width=12cm]{img/integ_drupal.png}
\caption{podešenje drupal easyrec modula}
\end{figure}

Nakon ovog podešenja, na easyrec web konzoli uočavamo da sistem bilježi "view" i "buy" akcije.

\chapter{Software toolset}
\begin{enumerate}
  \item Mac OS X 10.8.2
  \item mvim, vim tekst editor ver 7.3
  \item MacTex (TeX Live 2012)
\end{enumerate}

\chapter{Software repozitoriji}

\begin{itemize}
  \item drupal web content framework \url{https://github.com/hernad/drupal}
  \item easyrec source code \url{https://github.com/hernad/easyrec}
  \item PeB izvorni kod seminarskog rada \url{https://github.com/hernad/FIT\_PeB}

\end{itemize}

\end{document}
